\section{Bubble chamber tracks}
%http://teachers.web.cern.ch/teachers/archiv/HST2005/bubble_chambers/BCwebsite/index.htm
%http://epweb2.ph.bham.ac.uk/user/watkins/seeweb/BubbleChamber.htm
\label{bubble}

In this experiment, you will determine some properties of short-lived particles by analysing photographs from a liquid hydrogen bubble chamber.  The experiment is based around measurements of film taken from the bubble chamber at CERN particle physics laboratory near Geneva.

Bubble chambers played an important part in discovering particles whose existence played an important part in establishing the quark model in the 1950s--1970s. They are no longer in use at accelerator centres, having been superseded by the faster modern electronic detectors.  However, a bubble chamber is currently being used during the search for dark matter (WIMPs).

\subsection{Background}

The bubble chamber, invented by Donald Glaser in 1952, consists of a tank of unstable (superheated) transparent liquid.  Hydrogen is often used\footnote{Hydrogen was popular as it has the simplest nucleus; other nuclei presented problems such as, `Did the beam particle hit a neutron or a proton?'}, at a temperature of about \SI{30}{K}.  This liquid is very sensitive to the passage of {\bf charged} particles, which initiate boiling as a result of the energy they deposit by ionizing the atoms as they force their way through the liquid.
Bubbles are formed along the paths of the charged particles, and these tracks of bubbles can be photographed.

\subsection{How to read bubble chamber tracks}

\begin{itemize}
\item Beam particles whose tracks do NOT remain parallel all the way through the picture must have collided with a proton in a hydrogen atom.
\item All the paths of charged particles (and we only see charged particle paths) are curved by the magnetic field.
\item Positively charged particles' paths curve one way, negatively charged particles' paths the other.  (Normally, it is not necessary to be told which way the field is pointing---the picture already contains the answer. The little curly tracks are produced by electrons which are knocked out of the atoms by charged particles passing by.)
\item The momentum of a particle is proportional to the radius of curvature of the track in the bubble chamber. 
\item When a particle has used up its energy making bubbles, it stops. So its range is a measure of its energy.  In practice, this is useful for identifying protons that have received only a gentle blow from the beam particle, thus not having enough energy to get all the way to the edge of the chamber.
\item Other particles (for example pions) may stop; but if they do, they decay in a characteristic way which tells us that they are pions. In such a case, we would know the mass $m$ and its momentum $p$ from the curvature of the track; the energy can then be calculated using: $E^{2} = p^{2}c^{2} + m^{2}c^{4}$.
\end{itemize}

So much for charged particles.  Neutral particles do NOT leave trails of bubbles.  However, we can still sometimes glean some clues about their properties:

\begin{itemize}
\item An unstable neutral particle may decay before it leaves the bubble chamber into a pair of lighter particles - one positive and one negative - leaving an easily recognizable letter V (or vee) shape.
\item If the tracks from the vee happen to cross again, downstream, the line joining the crossing position to the decay position (the V) points back to the origin of the neutral particle.
\item An uncharged particle is often produced when an unstable charged particle decays--usually into a charged particle of the same sign and one or more neutral particles. This shows up in the bubble chamber as a kink (a sudden change into a more curved track).
\item The energy and momentum of the uncharged particle(s) which leave at the kink can be inferred by conservation laws from the tracks we do see.
\item If you want to examine the picture of a collision very carefully � to find small angle kinks, for example � it is a good idea to print the picture and look at it from a very low angle.
\end{itemize}

%It is often possible to identify the kind of particle from their characteristic decay signatures.  This is a case of visual pattern recognition.

%\includegraphics[width=4cm]{sig-proton.png}

%At the LHC, where the events will have hundreds of particles in the final state, imaginative systems of electronic detectors and software have been designed to do this at incredibly high data-acquisition rates.

\subsection{Event A}


\begin{questions}
\question
\begin{parts}
\part How many charged beam particles enter the chamber? \answerline
\part In what direction is the beam moving (from the bottom to the top or vice versa)? Draw arrows on the image to show this.
\part If there are any knock-on electrons, label these on the image.
%\part What is the direction of the magnetic field? \answerline
\part How many collisions do you see? \answerline
\part How many particles result from each of the collisions? \answerline
\part Identify the charges of these particles, and explain your reasoning. \fillwithlines{2cm}
\part What is the charge of the beam particles (assuming collisions are with protons)? \answerline
\part How many kinks do you see?\answerline
\part How many vees do you see? \answerline
\part How many particles result from the decays? \answerline
\part Identify the charges of the particles from decays, and explain your reasoning. \fillwithlines{2cm}
\part Consider the main collision/interaction. Which of the charged particles from the collision has the lowest momentum? How do you know?\fillwithlines{3cm}
\end{parts}

\includegraphics[width=0.7\textwidth]{img/event_Agal3_kzero11.png}\\
{\footnotesize Event A.  The crosses are known as fiducials and are marked at positions which are accurately surveyed. They are measured along with events and are essential to the 3D reconstruction process.} %(From the Latin word `fiducia' meaning truth.)}

\subsection{Event B}
\includegraphics[width=0.7\textwidth]{img/event_Bgal2_231.png}\\

\begin{parts}
\part In what direction is the beam going?  Mark this with an arrow on the photograph.
\part How many beam tracks are there? \answerline
\part How many collisions are there? \answerline
\part How many vees are there? \answerline 
\part How many kinks are there? \answerline
\part What does the fact that you see a track indicate? \fillwithlines{1cm}
\part How can you tell if two tracks represent particles of the opposite charge?\fillwithlines{2cm}
\part What physical property of a particle is determined by the curvature of its track?\answerline
\part What causes a kink?\fillwithlines{2cm}
\part What causes a vee?\fillwithlines{2cm}
\part What are the small crosses?\fillwithlines{1cm}
\part What is the charge of the target particles (remember this is a hydrogen bubble chamber)? \answerline
\part What is the charge of the beam and how do you know? \fillwithlines{1cm}
\part In which direction are the neutral particle(s) moving from the kinks? \fillwithlines{1cm}
\part What needs to happen for a track to kink twice? \fillwithlines{2cm}
%\part How does the difference in mass between the initial state (before the kink) and the final state (after the kink) affect the angle of the kink?
\part What particles can cause a vee? \fillwithlines{1cm}
\part From which point do the particles that cause the vees come? \fillwithlines{1cm}
\end{parts}


\subsection{Event C}
\begin{parts}
\part How many charged beam particles enter the chamber? \answerline
\part In what direction is the beam moving (from the bottom to the top or vice versa)?  Put an arrow on the photograph to show this.
\part Are there any knock-on electrons? If so, label these on the photograph.
%\part What is the direction of the magnetic field (assuming the targets are protons)?
\part How many collisions do you see? \answerline
\part How many particles result from each of the collisions? \fillwithlines{1cm}
\part Identify the charges of these particles.\fillwithlines{2cm}
\part What is the charge of the beam particles? How do you know? \fillwithlines{2cm}
\part How many kinks do you see? \answerline
\part How many vees do you see? \answerline
\part How many particles result from the decays? \fillwithlines{2cm}
\part Identify the charges of the particles from decays.\fillwithlines{3cm}
\part Identify the possibilities particle that decays, forming a vee.\fillwithlines{2cm}
\part Identify the possibilities for the particle that decays, forming a kink.\fillwithlines{2cm}
\end{parts}
\end{questions}
\includegraphics[width=0.7\textwidth]{img/event_Cgal3_omega_1.png}

