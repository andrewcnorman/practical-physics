\section{The Photoelectric effect}
\label{photoelectric}

In quantum theory, light exists as a stream of photons, each with energy $hf$, where $h$ is Planck's constant and $f$ is the frequency of the light.
\[E_{\text{photon}}=hf=\frac{hc}{\lambda},\]
where $\lambda$ is the wavelength of the light.

When the light is incident on a metal surface, an electron in the surface may absorb a photon, and therefore its energy.  The electron is held in the surface by an attractive force, which needs an amount of energy called the work function $\phi$ to be overcome.  It follows that an electron will be emitted if it absorbs a photon with energy greater then the work function.  Any excess energy will become kinetic energy of the electron.

This can be written in Einstein's photoelectric equation:
\[hf=\phi+E_{k}.\]

A (negative) voltage, called the stopping voltage $V_{S}$ can be applied to stop electrons leaving the surface.

\subsection{Setup}

In this practical, you will use the Virtual Physical Laboratory (VPL),\footnote{A software package written by Dr John Nunn of the National Physical Laboratory (NPL) and provided free to UK schools through the generous support of NPL and the Institute of Physics.} which is installed on the school laptops.  

When you open VPL, %instructions writeen for V10
you need to open the \texttt{Quantum > Photelectric effect} experiment, when you will be presented with the following:

\hfill \includegraphics[width=0.7\textwidth]{img/vpl-pe.png} \hfill{}

The apparatus in this virtual experiment consists of a metal surface (this might in practice be e.g.\ a photodiode, from which electrons will leave when light is incident) in a box with connexions, a digital voltmeter, to measure the stopping voltage $V_{S}$, a picoammeter to measure the very tiny current of the emitted electrons, and a set of filters.

\subsection{Measurements}
\begin{itemize}
\item Check that the reverse voltage is zero (drag the slider right down to the bottom), and the photoelectric current goes to zero when the lamp brightness is turned down to zero.
\item Set the lamp to about half the maximum brightness.
\item Make sure you are using target material 2 (select this using the yellow switch).  Select the violet filter (\SI{430}{nm}) and notice that you get a small photoelectric current ($\sim\SI{7}{pA}$) flowing.
\item Slowly increase the reverse voltage using the slider until the photoelectric current \emph{just} drops to zero.  This means that electrons are not leaving the surface.
\item Record this voltage $V_{S}$ in the table below. (NB You can also click \texttt{Store data} in the program which will allow you to plot a graph on the computer.  However, to get the most out of this experiment, only use this as a check at the end!)
\item The filters have different colours and so they each let a different wavelength through.  Copy down the wavelength and convert this into a frequency (in \SI{e14}{Hz}) using $f=\frac{c}{\lambda}$, where $c$ is the speed of light, \SI{3e8}{m.s^{-1}}.  The first one has been done for you.
\item Repeat this process for all the different coloured filters.
\end{itemize}

\begin{center}
\begin{tabular}{|c|c|p{2cm}|p{2cm}|}
\hline
Filter colour & $\lambda$/nm & \multicolumn{1}{|c|}{$f$/\SI{e14}{Hz}} & \multicolumn{1}{|c|}{$V_{S}$/V}\\
\hline
Infra-Red & & &\\
\hline
Red & & & \\
\hline
Orange & & & \\
\hline
Yellow & & & \\
\hline
Green & & & \\
\hline
Blue & & & \\
\hline
Indigo & & & \\
\hline
Violet & 430 & 6.98 & \\
\hline
Ultra-Violet & & & \\
\hline
\end{tabular}
\end{center}

\begin{questions}
\question Plot a graph of $V_{S}$ on the $y$-axis against frequency $f$ on the $x$-axis.  \emph{Choose your scales so that the line intercepts the $y$-axis.  This will mean going down to $-\SI{2.0}{V}$ and \SI{0.0e14}{Hz}}  Draw a line of best fit.

\newpage
\thispagestyle{empty}
\begin{tikzpicture}[remember picture, overlay]

\tikzset{normal lines/.style={gray, very thin}} 
\tikzset{margin lines/.style={gray, thick}} 
\tikzset{mm lines/.style={gray, ultra thin}} 
\tikzset{strong lines/.style={black, very thin}} 
\tikzset{master lines/.style={black, very thick}} 
\tikzset{dashed master lines/.style={loosely dashed, black, very thick}} 

\node at ([xshift=1cm, yshift=8.5mm] current page.south west){
  \begin{tikzpicture}[remember picture, overlay]

    \draw[style=mm lines,step=1mm] (0,0) grid +(19cm,28cm); 
    \draw[style=strong lines,step=1cm] (0,0) grid +(19cm,28cm); 

  \end{tikzpicture}
};
\end{tikzpicture}


\newpage
\thispagestyle{empty}
\begin{tikzpicture}[remember picture, overlay]

\tikzset{normal lines/.style={gray, very thin}} 
\tikzset{margin lines/.style={gray, thick}} 
\tikzset{mm lines/.style={gray, ultra thin}} 
\tikzset{strong lines/.style={black, very thin}} 
\tikzset{master lines/.style={black, very thick}} 
\tikzset{dashed master lines/.style={loosely dashed, black, very thick}} 

\node at ([xshift=1cm, yshift=8.5mm] current page.south west){
  \begin{tikzpicture}[remember picture, overlay]

    \draw[style=mm lines,step=1mm] (0,0) grid +(19cm,28cm); 
    \draw[style=strong lines,step=1cm] (0,0) grid +(19cm,28cm); 

  \end{tikzpicture}
};
\end{tikzpicture}

\newpage

\question Calculate and record the gradient of the graph, showing your working on the graph. \answerline

\question Measure and record the intercept of the line with the $y$-axis. \answerline

Rearranging the photoelectric equation,
\begin{align*}
hf&=\phi+E_{k},\\
hf&=\phi+eV_{S},\\
V_{S}&=\left(\frac{h}{e}\right)f-\frac{\phi}{e}.\\
\end{align*}
This final version of the equation is in the form $y=mx+c$.

\question Use this information and your previous answers to calculate
\begin{parts}
\part Planck's constant $h$,
\fillwithlines{1.5cm}
\part the work function $\phi$ of the target material,
\fillwithlines{1.5cm}
\part the work function into electron volts (eV).
\fillwithlines{1cm}
\end{parts}

\question If a material with a \emph{larger} work function were to be used, explain what effect, if any this would have on
\begin{parts}
\part the gradient,
\fillwithlines{2.5cm}
\part the intercept on the $y$-axis,
\fillwithlines{2.5cm}
NB If you have time, try using VPL to obtain and plot data for `taget material 1', which will allow you to test your predictions.
\end{parts}

\question The accepted value for Planck's constant is \SI{6.63e-34}{J.s}.  Calculate the percentage error in your measured value, showing your working
\fillwithlines{2cm}

\end{questions}
