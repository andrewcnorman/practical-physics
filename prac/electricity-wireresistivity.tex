\section{Resistivity of a wire}
\label{resistivity}

\begin{questions}
\question 
\begin{parts}
\part Using the micrometers, measure the diameter of the wire at three different points, and calculate $d$, the average diameter\\

\noindent\begin{tabular}{|p{2cm}|p{2cm}|p{2cm}|p{2cm}|}
\hline
$d_{1}/\si{mm}$ & $d_{1}/\si{mm}$ & $d_{1}/\si{mm}$ & $d/\si{mm}$\\
\hline
&&&\\
\hline
\end{tabular}\\
\part Calculate the average cross-sectional area of the wire, $A$, in \si{m^2}.
\fillwithlines{1cm}
\end{parts}
\question Set up the circuit as shown below, noting the following points:
\begin{itemize}
\item Set the length of the wire between the crocodile clips, $l$, to be \SI{1.000}{m}.  Make the wire as straight as possible.  It may be convenient to tape the wire to the desk.
\item Set the power supply voltage to about \SI{5}{V} d.c.
\item The ammeter should be on the \SI{10}{A} scale.
\item Adjust the ammeter so that the current, $I$, reads \SI{0.20}{A}.
\end{itemize}

\begin{center}
\begin{tikzpicture}[circuit ee IEC]
%bottom rail
\draw (0,0) node[contact]{} node[anchor=east]{\SI{0}{V}} to [circuit handle symbol={draw,shape=circle,label=center:{A},minimum size=10mm}] (8,0) to (11,0);
\draw (0,5) node{} node[anchor=east]{+\SI{5}{V}} node[contact]{} to [resistor=adjustable] (8,5) to (11,5);
%right end
\draw (11,0) to [circuit handle symbol={draw,shape=circle,label=center:{V},minimum size=10mm}] (11,5);
%contacts:
\draw (8,0) to (8,2) node[contact]{};
\draw (8,5) to (8,3) node[contact]{};
\draw (8,2.5) node {\SI{1.000}{m} wire};
\end{tikzpicture}
\end{center}

\begin{parts}
\part Record $V$, the voltage on the voltmeter \answerline

\part Calculate $R$, the resistance of the wire
\fillwithlines{1cm}
\end{parts}

\newpage

\question Repeat this proceduce for {\bf five} more lengths of wire.  \emph{If necessary, adjust the variable resistor so that the current through the wire remains at \SI{0.20}{A}.}\\
Record the results into the table below, including the result from the previous page.

\begin{center}
\begin{tabular}{|p{2.5cm}|p{2.5cm}|p{2.5cm}|}
\hline
\multicolumn{1}{|c|}{$l$/\si{m}} & \multicolumn{1}{|c|}{$V$/\si{V}} & \multicolumn{1}{|c|}{$R$/\si{\ohm}} \\
\hline
1.000 & &\\
\hline
& &\\
\hline
& &\\
\hline
& &\\
\hline
& &\\
\hline
& &\\
\hline
\end{tabular}
\end{center}

\question Theory tells us that
\[R=\frac{\rho l}{A},\]
where $\rho$ is the resistivity of the wire.\\

Therefore a graph of $R$ against $l$ should be a straight line of gradient $\frac{\rho}{A}$.
\begin{parts}

\part \emph{Plot a graph of $R$ against $l$.}

\part Calculate and record below $G$, the gradient of the line.
\fillwithlines{3cm}

\part Use the above information to calculate a value for $\rho$, the resistivity of the wire.
\fillwithlines{3cm}
\end{parts}

\newpage
\thispagestyle{empty}
\begin{tikzpicture}[remember picture, overlay]

\tikzset{normal lines/.style={gray, very thin}} 
\tikzset{margin lines/.style={gray, thick}} 
\tikzset{mm lines/.style={gray, ultra thin}} 
\tikzset{strong lines/.style={black, very thin}} 
\tikzset{master lines/.style={black, very thick}} 
\tikzset{dashed master lines/.style={loosely dashed, black, very thick}} 

\node at ([xshift=1cm, yshift=8.5mm] current page.south west){
  \begin{tikzpicture}[remember picture, overlay]

    \draw[style=mm lines,step=1mm] (0,0) grid +(19cm,28cm); 
    \draw[style=strong lines,step=1cm] (0,0) grid +(19cm,28cm); 

  \end{tikzpicture}
};
\end{tikzpicture}


\newpage
\thispagestyle{empty}
\begin{tikzpicture}[remember picture, overlay]

\tikzset{normal lines/.style={gray, very thin}} 
\tikzset{margin lines/.style={gray, thick}} 
\tikzset{mm lines/.style={gray, ultra thin}} 
\tikzset{strong lines/.style={black, very thin}} 
\tikzset{master lines/.style={black, very thick}} 
\tikzset{dashed master lines/.style={loosely dashed, black, very thick}} 

\node at ([xshift=1cm, yshift=8.5mm] current page.south west){
  \begin{tikzpicture}[remember picture, overlay]

    \draw[style=mm lines,step=1mm] (0,0) grid +(19cm,28cm); 
    \draw[style=strong lines,step=1cm] (0,0) grid +(19cm,28cm); 

  \end{tikzpicture}
};
\end{tikzpicture}

\newpage

\question
\begin{parts}

\part Explain why the current through the wire should be kept constant.
\fillwithlines{3cm}

\part What would be the effect on the graph if the current were to increase as the length of the wire is decreased?
\fillwithlines{3cm}

\part What is the uncertainty in the measurement of the length of the wire?
\fillwithlines{2cm}

\part What is the percentage uncertainty in the measurement of the length of the wire at
\begin{subparts}
\subpart \SI{1.000}{m}? \answerline
\subpart \SI{0.500}{m}? \answerline
\end{subparts}

\part Discuss the advantages and disadvantages of using a longer wire in making the experiment more reliable.
\fillwithlines{5cm}
\end{parts}
\end{questions}
