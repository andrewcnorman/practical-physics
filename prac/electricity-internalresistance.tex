\section{Internal Resistance}
\label{intres}

The challenge in this experiment is to determine the resistance of a small resistor which is hidden from view in a box.

\begin{center}
\begin{tikzpicture}[circuit ee IEC]
%power supply + rheostat slide
\draw (0,0) node[contact]{} node[anchor=east]{\SI{0}{V}} to (1,0) to [resistor] (1,5) to (0,5) node{} node[anchor=east]{+\SI{4}{V}} node[contact]{};
%bottom rail
\draw (0,0) to (11,0);
%right end
\draw (11,0) to [circuit handle symbol={draw,shape=circle,label=center:{V},minimum size=10mm}] (11,5);
%pot +top rail
\draw (2,2.5) to (2,5) to [circuit handle symbol={draw,shape=circle,label=center:{A},minimum size=10mm}] (8,5) to (11,5);
\draw[->] (2,2.5)--(1.2,2.5);
%contacts:
\draw (8,0) to (8,2) node[contact]{};
\draw (8,5) to (8,3) node[contact]{};
\draw (8,2.5) node {component};
\end{tikzpicture}
\end{center}

\begin{questions}
\question Using the variable resistor (and if necessary the voltage setting on the power supply), measure the current through  the ohmic conductor for voltages between \SI{0.00}{V} and \SI{0.60}{V}, ion \SI{0.10}{V} steps.

\begin{center}
\begin{tabular}{|c|c|}
\hline
Voltage $V$ / V & Current $I$ / mA\\
\hline 
& \\
\hline
& \\
\hline
& \\
\hline
& \\
\hline
& \\
\hline
& \\
\hline
& \\
\hline
\end{tabular}
\end{center}

\newpage

\question Replace the ohmic conductor with the filament lamp, and repeat the experiment.

\begin{center}
\begin{tabular}{|c|c|}
\hline
Voltage $V$ / V & Current $I$ / mA\\
\hline 
& \\
\hline
& \\
\hline
& \\
\hline
& \\
\hline
& \\
\hline
& \\
\hline
& \\
\hline
\end{tabular}
\end{center}


\question Replace the filament lamp withthe diode. \textbf{Make sure that the diode is forward biased}, i.e. it faces in the direction of conventional current flow.  Repeat the experiment, however this time increase the p.d. in steps of \SI{0.10}{V} until \SI{0.5}{V}, noting the current, then continue by increasing the current in steps of \SI{20.0}{mA}, noting the voltage.

\begin{center}
\begin{tabular}{|c|c|}
\hline
Voltage $V$ / V & Current $I$ / mA\\
\hline 
0.00 & \\
\hline
0.10 & \\
\hline
0.20 & \\
\hline
0.30 & \\
\hline
0.40 & \\
\hline
0.50 & \\
\hline
& 20.0\\
\hline
& 40.0\\
\hline
& 60.0\\
\hline
\end{tabular}
\end{center}

\question Plot a graph of $I$ on the $y$-axis against $V$ on the $x$-axis for each component \textbf{using the same scales (but different axes) for all.}


\newpage
\thispagestyle{empty}
\begin{tikzpicture}[remember picture, overlay]

\tikzset{normal lines/.style={gray, very thin}} 
\tikzset{margin lines/.style={gray, thick}} 
\tikzset{mm lines/.style={gray, ultra thin}} 
\tikzset{strong lines/.style={black, very thin}} 
\tikzset{master lines/.style={black, very thick}} 
\tikzset{dashed master lines/.style={loosely dashed, black, very thick}} 

\node at ([xshift=1cm, yshift=8.5mm] current page.south west){
  \begin{tikzpicture}[remember picture, overlay]

    \draw[style=mm lines,step=1mm] (0,0) grid +(19cm,28cm); 
    \draw[style=strong lines,step=1cm] (0,0) grid +(19cm,28cm); 

  \end{tikzpicture}
};
\end{tikzpicture}


\newpage
\thispagestyle{empty}
\begin{tikzpicture}[remember picture, overlay]

\tikzset{normal lines/.style={gray, very thin}} 
\tikzset{margin lines/.style={gray, thick}} 
\tikzset{mm lines/.style={gray, ultra thin}} 
\tikzset{strong lines/.style={black, very thin}} 
\tikzset{master lines/.style={black, very thick}} 
\tikzset{dashed master lines/.style={loosely dashed, black, very thick}} 

\node at ([xshift=1cm, yshift=8.5mm] current page.south west){
  \begin{tikzpicture}[remember picture, overlay]

    \draw[style=mm lines,step=1mm] (0,0) grid +(19cm,28cm); 
    \draw[style=strong lines,step=1cm] (0,0) grid +(19cm,28cm); 

  \end{tikzpicture}
};
\end{tikzpicture}

\newpage
\question \begin{parts} 
\part Calculate the gradient of the graph for the ohmic conductor

\fillwithdottedlines{3cm}

\part The resistance of the ohmic conductor is given by the reciprocal of this gradient.  Calculate the resistance of the ohmic conductor.

\fillwithdottedlines{2cm}

\part Using values from you table, calculate the resistance of the filament lamp for two different calues of voltage.

\fillwithdottedlines{2cm}

\part Explain why the resistance of the filament lamp changes in this way.

\fillwithdottedlines{3cm}

\part Explain the shape of the diode graph in terms of its resistance.

\fillwithdottedlines{3cm}

\part What would the graph for the diode look like if it were to be placed in the reverse direction?

\fillwithdottedlines{3cm}

\end{parts}

\end{questions}
