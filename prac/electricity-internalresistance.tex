\section{Internal Resistance}
\label{intres}
%NB this is closely based on the AQA June 2009 ISA

In this experiment you are going to carry out an experiment to investigate the relationship between the potential difference across the terminals of a \SI{1.5}{V} `AA' battery and the current through it.

You will use several different resistors, each of which will draw a different current through the battery.

\begin{questions}
\question Set up the circuit as show in the diagram below, using the \SI{22}{\ohm} resistor.
\begin{center}
\begin{tikzpicture}[circuit ee IEC]
%top rail with battery
\draw (0,5) to [battery] (8,5);
%left side and resistor
\draw (0,5) to (0,0) to [resistor] (8,0);
%right side with ammeter
\draw (8,0) to [circuit handle symbol={draw,shape=circle,label=center:{A},minimum size=10mm}] (8,5);
%voltmeter
\draw (2,5) to (2,3) to [circuit handle symbol={draw,shape=circle,label=center:{V},minimum size=10mm}] (6,3) to (6,5);
%labels
\draw (4,-0.5) node {\SI{22}{\ohm}};
\draw (4,5.5) node {\SI{1.5}{V} `AA' battery};
\end{tikzpicture}
\end{center}

\question Record the current shown on the ammeter. \answerline

\question What is the precision of the ammeter? \answerline

\question Using the instrument precision, work out the percentage uncertainty in the ammeter reading.
\fillwithlines{2cm}

\question Explain why using resistors with very high values would be unsuitable in this experiment.
\fillwithlines{3cm}

\newpage

\question Substituing each resistor into the circuit in turn---start by replacing the \SI{22}{\ohm} resistor with the \SI{10}{\ohm} resistor---record the values of the current and terminal p.d. in the table below.  \emph{Make sure you disconnect the battery between readings.}

\begin{center}
\begin{tabular}{|c|c|c|}
\hline
Resistor value / \si{\ohm} & Current $I$ / mA & Terminal p.d. / V\\
\hline 
22 & & \\
\hline
10 & & \\
\hline
6.8 & & \\
\hline
4.7 & & \\
\hline
3.3 & & \\
\hline
2.2 & & \\
\hline
1.5 & & \\
\hline
1.0 & & \\
\hline
\end{tabular}
\end{center}

\question Plot a graph of terminal p.d. on the $y$-axis against current $I$ on the $x$-axis.

The equation relating terminal p.d. $V$ and current $I$ is
\[V = \epsilon - Ir,\]
where $\epsilon$ is the emf of the supply and $r$ is the internal resistance of the supply.

\question By reference to the equation of a straight line $y = mx + c$,
\begin{parts}
\part what physical quantity is represented by the intercept on the p.d. axis?
\fillwithlines{1cm}
\part what physical quantity is represented by the gradient of the graph?
\fillwithlines{1cm}
\end{parts}

\question Use a line of best fit to help you to determine the emf and the internal resistance of the battery in this experiment, showing your working clearly.
\fillwithlines{5cm}

\newpage
\thispagestyle{empty}
\begin{tikzpicture}[remember picture, overlay]

\tikzset{normal lines/.style={gray, very thin}} 
\tikzset{margin lines/.style={gray, thick}} 
\tikzset{mm lines/.style={gray, ultra thin}} 
\tikzset{strong lines/.style={black, very thin}} 
\tikzset{master lines/.style={black, very thick}} 
\tikzset{dashed master lines/.style={loosely dashed, black, very thick}} 

\node at ([xshift=1cm, yshift=8.5mm] current page.south west){
  \begin{tikzpicture}[remember picture, overlay]

    \draw[style=mm lines,step=1mm] (0,0) grid +(19cm,28cm); 
    \draw[style=strong lines,step=1cm] (0,0) grid +(19cm,28cm); 

  \end{tikzpicture}
};
\end{tikzpicture}


\newpage
\thispagestyle{empty}
\begin{tikzpicture}[remember picture, overlay]

\tikzset{normal lines/.style={gray, very thin}} 
\tikzset{margin lines/.style={gray, thick}} 
\tikzset{mm lines/.style={gray, ultra thin}} 
\tikzset{strong lines/.style={black, very thin}} 
\tikzset{master lines/.style={black, very thick}} 
\tikzset{dashed master lines/.style={loosely dashed, black, very thick}} 

\node at ([xshift=1cm, yshift=8.5mm] current page.south west){
  \begin{tikzpicture}[remember picture, overlay]

    \draw[style=mm lines,step=1mm] (0,0) grid +(19cm,28cm); 
    \draw[style=strong lines,step=1cm] (0,0) grid +(19cm,28cm); 

  \end{tikzpicture}
};
\end{tikzpicture}

\newpage
\question Why do you think you were instructed to switch off or disconnect the battery between readings?
\fillwithlines{2cm}

\question Do you think your readings are reliable? Give a reason for your answer.
\fillwithlines{2cm}

\question The manufacturer quotes the resistors used as having an uncertainty (the
manufacturer's `tolerance') of 5\% . 
\begin{parts}
\part Calculate the maximum possible value of the \SI{6.8}{\ohm} resistor used in this experiment. \answerline
\part Explain why it would not have made any difference to the value of  obtained in the
experiment if resistors with a tolerance of only 2\%\ had been used instead.
\fillwithlines{2cm}
\end{parts}

\question The voltmeter used in the above experiment was found to have a calibration error
whereby every reading was \SI{0.22}{V} too high.
\begin{parts}
\part What is the name given to this type of error?
\fillwithlines{1cm}
\part How, if at all, would this have affected the value obtained for the intercept of the graph?
\fillwithlines{2cm}
\part How, if at all, would this have affected the value for the gradient of the graph?
\fillwithlines{2cm}
\end{parts}

\end{questions}
