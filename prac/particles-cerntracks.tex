\section{Particle tracks}
%http://teachers.web.cern.ch/teachers/archiv/HST2005/bubble_chambers/BCwebsite/index.htm
%http://epweb2.ph.bham.ac.uk/user/watkins/seeweb/BubbleChamber.htm
\label{tracks}

In this experiment, you will use an app for iPad developed by the University of Oxford to identify events in the ATLAS experiment of the LHC (Large Hadron Collider) at CERN particle physics laboratory near Geneva.

\subsubsection{Background}
In the LHC, protons are accelerated around a ring \SI{27}{km} in circumference at speeds which are incredibly close to the speed of light.  They are then collided head on with protons travelling in the opposite direction in tremendously energetic collisions, which have enough energy to probe matter at a very small length scale and produce new particles in interaction events.  The collisions are arranged so that they occur at one of four points in the collider, where an experiment has been built to investigate the events which occur in detail.  These four main experiments are ATLAS, ALICE, LHCb and CMS.

Events are identified by analysing the instrumentation of the detectors surrounding the collision site.  There are several different layers of detectors inside ATLAS, each of which will detect a different type of particle.  A new particle created in the high energy collision where the beams of protons collide will travel out through the different layers of ATLAS, and this gives it the chance to interact with the detectors.  A particle can be identified by the way in which it interacts with the various detectors in ATLAS.  The computers will display the tracks of the particles, and label them depending on which detectors they activate.  This allows scientists to distinguish different kinds of event, by looking at the particles produced, and hunt for clues which suggest new physics.

\subsection{Game 1: Tutorial (\Pelectron, \Pmu, \Pneutrino)}

You are now ready to play the first `game' in the Collider app.  In this tutorial, you will need to use the information on each particle to help you to identify it.  

\subsubsection{Identifying particles in Collider}

\begin{minipage}[t]{0.25\textwidth}
\begin{center}
\includegraphics[width=0.8\textwidth]{img/collider-eg1-electron.png}
\end{center}
Electrons have a track and deposit energy in the electromagnetic calorimeter. Collider will draw a cone around recognised electrons.
\end{minipage}~
\begin{minipage}[t]{0.25\textwidth}
\begin{center}
\includegraphics[width=0.8\textwidth]{img/collider-eg2-muon.png}
\end{center}
Muons have a track extending from the inner detector to the muon spectrometer. Muons may leave small energy deposits in the calorimeter.
\end{minipage}~
\begin{minipage}[t]{0.25\textwidth}
\begin{center}
\includegraphics[width=0.8\textwidth]{img/collider-eg3-jet.png}
\end{center}
Jets contain multiple tracks leading to varying amounts of energy deposited in both calorimeters. Jets may be produced by quarks or gluons.
\end{minipage}~
\begin{minipage}[t]{0.25\textwidth}
\begin{center}
\includegraphics[width=0.8\textwidth]{img/collider-eg4-neutrino.png}
\end{center}
Neutrinos cannot be detected in ATLAS.  However, by looking for missing energy---calculated by the computers and shown as a pink arrow in Collider---we can infer when they are produced.
\end{minipage}

\renewcommand{\labelenumi}{\Alph{enumi}: }

\subsubsection{Your answers}
As you play, and select your responses on screen, they will appear green or red---in this first round only---to give you instant feedback on your choices.  This should help you to circle the correct answers below.\\

%ANSWERS: 1A, 2B, 3C, 4D
\noindent\begin{minipage}[t]{0.25\textwidth}
\begin{center}
\includegraphics[width=\textwidth]{img/collider-l1-1.png}
\end{center}
\begin{enumerate}
\item Electron (\Pelectron)
\item Muon (\Pmuon)
\item Jet
\item Neutrino (\Pneutrino)
\end{enumerate}
\end{minipage}~
\begin{minipage}[t]{0.25\textwidth}
\begin{center}
\includegraphics[width=\textwidth]{img/collider-l1-2.png}
\end{center}
\begin{enumerate}
\item Electron (\Pelectron)
\item Muon (\Pmuon)
\item Jet
\item Neutrino (\Pneutrino)
\end{enumerate}
\end{minipage}~
\begin{minipage}[t]{0.25\textwidth}
\begin{center}
\includegraphics[width=\textwidth]{img/collider-l1-3.png}
\end{center}
\begin{enumerate}
\item Electron (\Pelectron)
\item Muon (\Pmuon)
\item Jet
\item Neutrino (\Pneutrino)
\end{enumerate}
\end{minipage}~
\begin{minipage}[t]{0.25\textwidth}
\begin{center}
\includegraphics[width=\textwidth]{img/collider-l1-4.png}
\end{center}
\begin{enumerate}
\item Electron (\Pelectron)
\item Muon (\Pmuon)
\item Jet
\item Neutrino (\Pneutrino)
\end{enumerate}
\end{minipage}

\subsection{Game 2: W/Z Level 1 (W, Z)}

The remaining games all require you to reconstruct events based on what you can see in the Collider interface, based on the ATLAS detector signals fed into the computers at CERN.  Each event may contain particles which decay into other particles. It becomes necessary to put together several pieces to see the overall picture.

\subsubsection{Detecting a W Boson}

A W boson can decay to a charged lepton and a neutrino. The charged lepton could be either an electron, muon or tau particle. Since tau particles decay quickly, it is easiest to look for an electron or a muon, together with a large amount of missing energy carried by the neutrino.

An example W boson event can be seen below. The short-lived W boson always decays before it can be seen -- this one has decayed into an electron (green cone) and an invisible neutrino (pink arrow).

\begin{center}
\includegraphics[width=0.4\textwidth]{img/collider-r1-wenu.png}
\end{center}

\subsubsection{Detecting a Z Boson}

The Z boson can decay to quarks or to any lepton together with its anti-particle. The easiest way to spot a Z boson is when it decays to an electron together with its anti-particle, or to a muon together with its anti-particle.

An example Z boson event can be seen below. This Z boson has decayed into a muon and an anti-muon (red lines).

\begin{center}
\includegraphics[width=0.4\textwidth]{img/collider-r2-zmumu.png}
\end{center}

\subsubsection{The four kinds of event}

There are eight events in this game, each of which is a W or Z boson undergoing one of four kinds of decay process:\\

\noindent\begin{minipage}[t]{0.25\textwidth}
\begin{center}
W to muon+neutrino
\[\PWminus\longrightarrow\Pmuon\Pnum\]
\[(\text{or }\PWplus\longrightarrow\APmuon\APnum)\]
\begin{tikzpicture}
\draw[style={boson}]
  (0, 0) -- node[above] {\PWminus} (2,0) ;
  \draw[style={electron}]
  (2, 0) -- node[right] {\Pmuon}(3, 2) ;
  \draw[style={electron}]
   (2, 0)  -- node[right] {\Pnum}(3, -2) ;
\end{tikzpicture}
\end{center}
\end{minipage}
\begin{minipage}[t]{0.25\textwidth}
\begin{center}
{\footnotesize W to electron+neutrino}
\[\PWminus\longrightarrow\Pelectron\Pnue\]
\[(\text{or }\PWplus\longrightarrow\APelectron\APnue)\]
\begin{tikzpicture}
\draw[style={boson}]
  (0, 0) -- node[above] {\PWminus} (2,0) ;
  \draw[style={electron}]
  (2, 0) -- node[right] {\Pelectron}(3, 2) ;
  \draw[style={electron}]
   (2, 0)  -- node[right] {\Pnue}(3, -2) ;
\end{tikzpicture}
\end{center}
\end{minipage}
\begin{minipage}[t]{0.25\textwidth}
\begin{center}
Z to electron+positron
\[\PZzero\longrightarrow\Pelectron\APelectron\]
\[\;\]
\begin{tikzpicture}
\draw[style={boson}]
  (0, 0) -- node[above] {\PZzero} (2,0) ;
  \draw[style={electron}]
  (2, 0) -- node[right] {\Pelectron}(3, 2) ;
  \draw[style={electron}]
   (2, 0)  -- node[right] {\APelectron}(3, -2) ;
\end{tikzpicture}
\end{center}
\end{minipage}
\begin{minipage}[t]{0.25\textwidth}
\begin{center}
Z to muon+antimuon
\[\PZzero\longrightarrow\Pmuon\APmuon\]
\[\;\]
\begin{tikzpicture}
\draw[style={boson}]
  (0, 0) -- node[above] {\PZzero} (2,0) ;
  \draw[style={electron}]
  (2, 0) -- node[right] {\Pmuon}(3, 2) ;
  \draw[style={electron}]
   (2, 0)  -- node[right] {\APmuon}(3, -2) ;
\end{tikzpicture}
\end{center}
\end{minipage}

\subsubsection{Your answers}

% ANSWERS: 1A, 2B, 3A, 4C, 5B, 6A, 7B, 8D
\noindent\begin{minipage}[t]{0.25\textwidth}
\begin{center}
Event 1/8\\
\includegraphics[width=\textwidth]{img/collider-l2-1.png}
\end{center}
\begin{enumerate}
\item $\PWminus\longrightarrow\Pmuon\Pnum$
\item $\PWminus\longrightarrow\Pelectron\Pnue$
\item $\PZzero\longrightarrow\Pelectron\APelectron$
\item $\PZzero\longrightarrow\Pmuon\APmuon$
\end{enumerate}
\end{minipage}~
\begin{minipage}[t]{0.25\textwidth}
\begin{center}
Event 2/8\\
\includegraphics[width=\textwidth]{img/collider-l2-2.png}
\end{center}
\begin{enumerate}
\item $\PWminus\longrightarrow\Pmuon\Pnum$
\item $\PWminus\longrightarrow\Pelectron\Pnue$
\item $\PZzero\longrightarrow\Pelectron\APelectron$
\item $\PZzero\longrightarrow\Pmuon\APmuon$
\end{enumerate}
\end{minipage}~
\begin{minipage}[t]{0.25\textwidth}
\begin{center}
Event 3/8\\
\includegraphics[width=\textwidth]{img/collider-l2-3.png}
\end{center}
\begin{enumerate}
\item $\PWminus\longrightarrow\Pmuon\Pnum$
\item $\PWminus\longrightarrow\Pelectron\Pnue$
\item $\PZzero\longrightarrow\Pelectron\APelectron$
\item $\PZzero\longrightarrow\Pmuon\APmuon$
\end{enumerate}
\end{minipage}~
\begin{minipage}[t]{0.25\textwidth}
\begin{center}
Event 4/8\\
\includegraphics[width=\textwidth]{img/collider-l2-4.png}
\end{center}
\begin{enumerate}
\item $\PWminus\longrightarrow\Pmuon\Pnum$
\item $\PWminus\longrightarrow\Pelectron\Pnue$
\item $\PZzero\longrightarrow\Pelectron\APelectron$
\item $\PZzero\longrightarrow\Pmuon\APmuon$
\end{enumerate}
\end{minipage}

\noindent\begin{minipage}[t]{0.25\textwidth}
\begin{center}
Event 5/8\\
\includegraphics[width=\textwidth]{img/collider-l2-5.png}
\end{center}
\begin{enumerate}
\item $\PWminus\longrightarrow\Pmuon\Pnum$
\item $\PWminus\longrightarrow\Pelectron\Pnue$
\item $\PZzero\longrightarrow\Pelectron\APelectron$
\item $\PZzero\longrightarrow\Pmuon\APmuon$
\end{enumerate}
\end{minipage}~
\begin{minipage}[t]{0.25\textwidth}
\begin{center}
Event 6/8\\
\includegraphics[width=\textwidth]{img/collider-l2-6.png}
\end{center}
\begin{enumerate}
\item $\PWminus\longrightarrow\Pmuon\Pnum$
\item $\PWminus\longrightarrow\Pelectron\Pnue$
\item $\PZzero\longrightarrow\Pelectron\APelectron$
\item $\PZzero\longrightarrow\Pmuon\APmuon$
\end{enumerate}
\end{minipage}~
\begin{minipage}[t]{0.25\textwidth}
\begin{center}
Event 7/8\\
\includegraphics[width=\textwidth]{img/collider-l2-7.png}
\end{center}
\begin{enumerate}
\item $\PWminus\longrightarrow\Pmuon\Pnum$
\item $\PWminus\longrightarrow\Pelectron\Pnue$
\item $\PZzero\longrightarrow\Pelectron\APelectron$
\item $\PZzero\longrightarrow\Pmuon\APmuon$
\end{enumerate}
\end{minipage}~
\begin{minipage}[t]{0.25\textwidth}
\begin{center}
Event 8/8\\
\includegraphics[width=\textwidth]{img/collider-l2-8.png}
\end{center}
\begin{enumerate}
\item $\PWminus\longrightarrow\Pmuon\Pnum$
\item $\PWminus\longrightarrow\Pelectron\Pnue$
\item $\PZzero\longrightarrow\Pelectron\APelectron$
\item $\PZzero\longrightarrow\Pmuon\APmuon$
\end{enumerate}
\end{minipage}

\begin{questions}
\setcounter{question}{8}
\question Look again at the last event (8/8).  Why do you think there is a pink arrow shown on the display, even though no neutrino ought to be produced in this event?

\fillwithlines{2cm}

\question Can you draw a Feynman diagram for $\PWplus\longrightarrow\APelectron\APnue$ in the space below?

\hfill

\end{questions}

\subsection{Game 3: W/Z Level 2} %12

% ANSWERS: 1A, 2B, 3B, 4C, 5B, 6C, 7C, 8B, 9B, 10C, 11D, 12C
\noindent\begin{minipage}[t]{0.25\textwidth}
\begin{center}
Event 1/12\\
\includegraphics[width=\textwidth]{img/collider-l3-1.png}
\end{center}
\begin{enumerate}
\item $\PZzero\longrightarrow\Pmuon\APmuon$
\item $\PWminus\longrightarrow\Pelectron\Pnue$
\item $\PWminus\longrightarrow\Pmuon\Pnum$
\item $\PZzero\longrightarrow\Pelectron\APelectron$
\end{enumerate}
\end{minipage}~
\begin{minipage}[t]{0.25\textwidth}
\begin{center}
Event 2/12\\
\includegraphics[width=\textwidth]{img/collider-l3-2.png}
\end{center}
\begin{enumerate}
\item $\PZzero\longrightarrow\Pmuon\APmuon$
\item $\PWminus\longrightarrow\Pelectron\Pnue$
\item $\PWminus\longrightarrow\Pmuon\Pnum$
\item $\PZzero\longrightarrow\Pelectron\APelectron$
\end{enumerate}
\end{minipage}~
\begin{minipage}[t]{0.25\textwidth}
\begin{center}
Event 3/12\\
\includegraphics[width=\textwidth]{img/collider-l3-3.png}
\end{center}
\begin{enumerate}
\item $\PZzero\longrightarrow\Pmuon\APmuon$
\item $\PWminus\longrightarrow\Pelectron\Pnue$
\item $\PWminus\longrightarrow\Pmuon\Pnum$
\item $\PZzero\longrightarrow\Pelectron\APelectron$
\end{enumerate}
\end{minipage}~
\begin{minipage}[t]{0.25\textwidth}
\begin{center}
Event 4/12\\
\includegraphics[width=\textwidth]{img/collider-l3-4.png}
\end{center}
\begin{enumerate}
\item $\PZzero\longrightarrow\Pmuon\APmuon$
\item $\PWminus\longrightarrow\Pelectron\Pnue$
\item $\PWminus\longrightarrow\Pmuon\Pnum$
\item $\PZzero\longrightarrow\Pelectron\APelectron$
\end{enumerate}
\end{minipage}

\vfill

\noindent\begin{minipage}[t]{0.25\textwidth}
\begin{center}
Event 5/12\\
\includegraphics[width=\textwidth]{img/collider-l3-5.png}
\end{center}
\begin{enumerate}
\item $\PZzero\longrightarrow\Pmuon\APmuon$
\item $\PWminus\longrightarrow\Pelectron\Pnue$
\item $\PWminus\longrightarrow\Pmuon\Pnum$
\item $\PZzero\longrightarrow\Pelectron\APelectron$
\end{enumerate}
\end{minipage}~
\begin{minipage}[t]{0.25\textwidth}
\begin{center}
Event 6/12\\
\includegraphics[width=\textwidth]{img/collider-l3-6.png}
\end{center}
\begin{enumerate}
\item $\PZzero\longrightarrow\Pmuon\APmuon$
\item $\PWminus\longrightarrow\Pelectron\Pnue$
\item $\PWminus\longrightarrow\Pmuon\Pnum$
\item $\PZzero\longrightarrow\Pelectron\APelectron$
\end{enumerate}
\end{minipage}~
\begin{minipage}[t]{0.25\textwidth}
\begin{center}
Event 7/12\\
\includegraphics[width=\textwidth]{img/collider-l3-7.png}
\end{center}
\begin{enumerate}
\item $\PZzero\longrightarrow\Pmuon\APmuon$
\item $\PWminus\longrightarrow\Pelectron\Pnue$
\item $\PWminus\longrightarrow\Pmuon\Pnum$
\item $\PZzero\longrightarrow\Pelectron\APelectron$
\end{enumerate}
\end{minipage}~
\begin{minipage}[t]{0.25\textwidth}
\begin{center}
Event 8/12\\
\includegraphics[width=\textwidth]{img/collider-l3-8.png}
\end{center}
\begin{enumerate}
\item $\PZzero\longrightarrow\Pmuon\APmuon$
\item $\PWminus\longrightarrow\Pelectron\Pnue$
\item $\PWminus\longrightarrow\Pmuon\Pnum$
\item $\PZzero\longrightarrow\Pelectron\APelectron$
\end{enumerate}
\end{minipage}

\vfill

\noindent\begin{minipage}[t]{0.25\textwidth}
\begin{center}
Event 9/12\\
\includegraphics[width=\textwidth]{img/collider-l3-9.png}
\end{center}
\begin{enumerate}
\item $\PZzero\longrightarrow\Pmuon\APmuon$
\item $\PWminus\longrightarrow\Pelectron\Pnue$
\item $\PWminus\longrightarrow\Pmuon\Pnum$
\item $\PZzero\longrightarrow\Pelectron\APelectron$
\end{enumerate}
\end{minipage}~
\begin{minipage}[t]{0.25\textwidth}
\begin{center}
Event 10/12\\
\includegraphics[width=\textwidth]{img/collider-l3-10.png}
\end{center}
\begin{enumerate}
\item $\PZzero\longrightarrow\Pmuon\APmuon$
\item $\PWminus\longrightarrow\Pelectron\Pnue$
\item $\PWminus\longrightarrow\Pmuon\Pnum$
\item $\PZzero\longrightarrow\Pelectron\APelectron$
\end{enumerate}
\end{minipage}~
\begin{minipage}[t]{0.25\textwidth}
\begin{center}
Event 11/12\\
\includegraphics[width=\textwidth]{img/collider-l3-11.png}
\end{center}
\begin{enumerate}
\item $\PZzero\longrightarrow\Pmuon\APmuon$
\item $\PWminus\longrightarrow\Pelectron\Pnue$
\item $\PWminus\longrightarrow\Pmuon\Pnum$
\item $\PZzero\longrightarrow\Pelectron\APelectron$
\end{enumerate}
\end{minipage}~
\begin{minipage}[t]{0.25\textwidth}
\begin{center}
Event 12/12\\
\includegraphics[width=\textwidth]{img/collider-l3-12.png}
\end{center}
\begin{enumerate}
\item $\PZzero\longrightarrow\Pmuon\APmuon$
\item $\PWminus\longrightarrow\Pelectron\Pnue$
\item $\PWminus\longrightarrow\Pmuon\Pnum$
\item $\PZzero\longrightarrow\Pelectron\APelectron$
\end{enumerate}
\end{minipage}

\begin{questions}
\setcounter{question}{12}
\question Count up the number of answers for each option.

\begin{center}
\renewcommand{\arraystretch}{1.6}
\begin{tabular}{|c|c|c|c|}
\hline
A & B & C & D \\
\hline
$\PZzero\longrightarrow\Pmuon\APmuon$ & $\PWminus\longrightarrow\Pelectron\Pnue$ & $\PWminus\longrightarrow\Pmuon\Pnum$ & $\PZzero\longrightarrow\Pelectron\APelectron$ \\
\hline
& & & \\
\hline
\end{tabular}
\renewcommand{\arraystretch}{1}
\end{center}


\question What do you notice about the number of \PZzero particles detected compared to the number of \PWpm?  Can you suggest a reason for this?

\fillwithlines{2cm}

\end{questions}

\subsection{Game 4: W/Z Level 3} %15

In this game, there is an extra option: `Background'.  Every second, around 1 billion collisions occur in the centre of the ATLAS detector, and very many of them will be uninteresting collisions or background.  For this reason, ATLAS uses an advanced ��trigger�� system to tell the detector which events to record and which to ignore.  Even so, ATLAS records around 3,200 TeraBytes of data each year, equivalent to around 600 years of music.

\subsubsection{Your answers}

% ANSWERS: 1A, 2B, 3C, 4A, 5D, 6B, 7B, 8A, 9D, 10B, 11E, 12A, 13B, 14D, 15A
\noindent\begin{minipage}[t]{0.25\textwidth}
\begin{center}
Event 1/15\\
\includegraphics[width=\textwidth]{img/collider-l4-1.png}
\end{center}
\begin{enumerate}
\item $\PWminus\longrightarrow\Pmuon\Pnum$
\item $\PWminus\longrightarrow\Pelectron\Pnue$
\item $\PZzero\longrightarrow\Pelectron\APelectron$
\item Background
\item $\PZzero\longrightarrow\Pmuon\APmuon$
\end{enumerate}
\end{minipage}~
\begin{minipage}[t]{0.25\textwidth}
\begin{center}
Event 2/15\\
\includegraphics[width=\textwidth]{img/collider-l4-2.png}
\end{center}
\begin{enumerate}
\item $\PWminus\longrightarrow\Pmuon\Pnum$
\item $\PWminus\longrightarrow\Pelectron\Pnue$
\item $\PZzero\longrightarrow\Pelectron\APelectron$
\item Background
\item $\PZzero\longrightarrow\Pmuon\APmuon$
\end{enumerate}
\end{minipage}~
\begin{minipage}[t]{0.25\textwidth}
\begin{center}
Event 3/15\\
\includegraphics[width=\textwidth]{img/collider-l4-3.png}
\end{center}
\begin{enumerate}
\item $\PWminus\longrightarrow\Pmuon\Pnum$
\item $\PWminus\longrightarrow\Pelectron\Pnue$
\item $\PZzero\longrightarrow\Pelectron\APelectron$
\item Background
\item $\PZzero\longrightarrow\Pmuon\APmuon$
\end{enumerate}
\end{minipage}~
\begin{minipage}[t]{0.25\textwidth}
\begin{center}
Event 4/15\\
\includegraphics[width=\textwidth]{img/collider-l4-4.png}
\end{center}
\begin{enumerate}
\item $\PWminus\longrightarrow\Pmuon\Pnum$
\item $\PWminus\longrightarrow\Pelectron\Pnue$
\item $\PZzero\longrightarrow\Pelectron\APelectron$
\item Background
\item $\PZzero\longrightarrow\Pmuon\APmuon$
\end{enumerate}
\end{minipage}

\vfill

\noindent\begin{minipage}[t]{0.25\textwidth}
\begin{center}
Event 5/15\\
\includegraphics[width=\textwidth]{img/collider-l4-5.png}
\end{center}
\begin{enumerate}
\item $\PWminus\longrightarrow\Pmuon\Pnum$
\item $\PWminus\longrightarrow\Pelectron\Pnue$
\item $\PZzero\longrightarrow\Pelectron\APelectron$
\item Background
\item $\PZzero\longrightarrow\Pmuon\APmuon$
\end{enumerate}
\end{minipage}~
\begin{minipage}[t]{0.25\textwidth}
\begin{center}
Event 6/15\\
\includegraphics[width=\textwidth]{img/collider-l4-6.png}
\end{center}
\begin{enumerate}
\item $\PWminus\longrightarrow\Pmuon\Pnum$
\item $\PWminus\longrightarrow\Pelectron\Pnue$
\item $\PZzero\longrightarrow\Pelectron\APelectron$
\item Background
\item $\PZzero\longrightarrow\Pmuon\APmuon$
\end{enumerate}
\end{minipage}~
\begin{minipage}[t]{0.25\textwidth}
\begin{center}
Event 7/15\\
\includegraphics[width=\textwidth]{img/collider-l4-7.png}
\end{center}
\begin{enumerate}
\item $\PWminus\longrightarrow\Pmuon\Pnum$
\item $\PWminus\longrightarrow\Pelectron\Pnue$
\item $\PZzero\longrightarrow\Pelectron\APelectron$
\item Background
\item $\PZzero\longrightarrow\Pmuon\APmuon$
\end{enumerate}
\end{minipage}~
\begin{minipage}[t]{0.25\textwidth}
\begin{center}
Event 8/15\\
\includegraphics[width=\textwidth]{img/collider-l4-8.png}
\end{center}
\begin{enumerate}
\item $\PWminus\longrightarrow\Pmuon\Pnum$
\item $\PWminus\longrightarrow\Pelectron\Pnue$
\item $\PZzero\longrightarrow\Pelectron\APelectron$
\item Background
\item $\PZzero\longrightarrow\Pmuon\APmuon$
\end{enumerate}
\end{minipage}

\vfill

\noindent\begin{minipage}[t]{0.25\textwidth}
\begin{center}
Event 9/15\\
\includegraphics[width=\textwidth]{img/collider-l4-9.png}
\end{center}
\begin{enumerate}
\item $\PWminus\longrightarrow\Pmuon\Pnum$
\item $\PWminus\longrightarrow\Pelectron\Pnue$
\item $\PZzero\longrightarrow\Pelectron\APelectron$
\item Background
\item $\PZzero\longrightarrow\Pmuon\APmuon$
\end{enumerate}
\end{minipage}~
\begin{minipage}[t]{0.25\textwidth}
\begin{center}
Event 10/15\\
\includegraphics[width=\textwidth]{img/collider-l4-10.png}
\end{center}
\begin{enumerate}
\item $\PWminus\longrightarrow\Pmuon\Pnum$
\item $\PWminus\longrightarrow\Pelectron\Pnue$
\item $\PZzero\longrightarrow\Pelectron\APelectron$
\item Background
\item $\PZzero\longrightarrow\Pmuon\APmuon$
\end{enumerate}
\end{minipage}~
\begin{minipage}[t]{0.25\textwidth}
\begin{center}
Event 11/15\\
\includegraphics[width=\textwidth]{img/collider-l4-11.png}
\end{center}
\begin{enumerate}
\item $\PWminus\longrightarrow\Pmuon\Pnum$
\item $\PWminus\longrightarrow\Pelectron\Pnue$
\item $\PZzero\longrightarrow\Pelectron\APelectron$
\item Background
\item $\PZzero\longrightarrow\Pmuon\APmuon$
\end{enumerate}
\end{minipage}~
\begin{minipage}[t]{0.25\textwidth}
\begin{center}
Event 12/15\\
\includegraphics[width=\textwidth]{img/collider-l4-12.png}
\end{center}
\begin{enumerate}
\item $\PWminus\longrightarrow\Pmuon\Pnum$
\item $\PWminus\longrightarrow\Pelectron\Pnue$
\item $\PZzero\longrightarrow\Pelectron\APelectron$
\item Background
\item $\PZzero\longrightarrow\Pmuon\APmuon$
\end{enumerate}
\end{minipage}

\vfill

\noindent\begin{minipage}[t]{0.25\textwidth}
\begin{center}
Event 13/15\\
\includegraphics[width=\textwidth]{img/collider-l4-13.png}
\end{center}
\begin{enumerate}
\item $\PWminus\longrightarrow\Pmuon\Pnum$
\item $\PWminus\longrightarrow\Pelectron\Pnue$
\item $\PZzero\longrightarrow\Pelectron\APelectron$
\item Background
\item $\PZzero\longrightarrow\Pmuon\APmuon$
\end{enumerate}
\end{minipage}~
\begin{minipage}[t]{0.25\textwidth}
\begin{center}
Event 14/15\\
\includegraphics[width=\textwidth]{img/collider-l4-14.png}
\end{center}
\begin{enumerate}
\item $\PWminus\longrightarrow\Pmuon\Pnum$
\item $\PWminus\longrightarrow\Pelectron\Pnue$
\item $\PZzero\longrightarrow\Pelectron\APelectron$
\item Background
\item $\PZzero\longrightarrow\Pmuon\APmuon$
\end{enumerate}
\end{minipage}~
\begin{minipage}[t]{0.25\textwidth}
\begin{center}
Event 15/15\\
\includegraphics[width=\textwidth]{img/collider-l4-15.png}
\end{center}
\begin{enumerate}
\item $\PWminus\longrightarrow\Pmuon\Pnum$
\item $\PWminus\longrightarrow\Pelectron\Pnue$
\item $\PZzero\longrightarrow\Pelectron\APelectron$
\item Background
\item $\PZzero\longrightarrow\Pmuon\APmuon$
\end{enumerate}
\end{minipage}

\begin{questions}
\setcounter{question}{15}
\question What percentage of the events in this round were background events?
\answerline

\question If these events were accidentally included by the ATLAS software, what kind of error (random or systematic) might this lead to in the results?  Explain your reasoning.

\fillwithlines{2cm}

\end{questions}

\subsection{Game 5: Hunt the Higgs} %20

In this final game, you have the chance to `detect' a Higgs boson, by looking for its distinctive decay signature.  

\subsubsection{The Higgs boson}
It is one of the deep insights of particle physics that most particles don't have any intrinsic mass at all. The Standard Model tells us that all of the fundamental particles, from the electron to the W boson, ought to pick up their mass by interacting with a field that surrounds them. This `Higgs' field must exist everywhere in the universe, according to the theory, but showing its existence directly is rather challenging.

In 1964 Peter Higgs noted that this field could be excited, and that this excitation of the field would appear as a new type of particle: a so-called `Higgs boson' particle.  If the new particle exists, then it should be produced in high-energy collisions. Part of the energy of the collision is converted into the mass of the new particle, according to Einstein's famous formula $E=mc^{2}$.

It was more than 50 years before the technology developed which allowed this prediction to be tested in the LHC at CERN.  The discovery of a new particle with mass $\sim\SI{125}{GeV/c^{2}}$, consistent with the one predicted by Professor Higgs, was announced in July 2012 in two papers by the ATLAS and CMS collaborations.

\subsubsection{Detecting a Higgs boson}

The Higgs boson can decay (its lifetime is only \SI{1.6e-22}{s}) into pairs of quarks, leptons or gauge bosons.  As general rule, the Higgs is more likely to decay into heavy particles than light particles, because the mass of a particle is proportional to the strength of its interaction with the Higgs.  An easy way to spot a Higgs is when it decays to two Z bosons, each of which susequently decays into a pair of leptons (electrons or muons).  Although only 3\% of Higgs bosons decay this way, this `Higgs to 4 leptons' signature is what we shall look for.

An example Higgs boson event can be seen below (in an \SI{8}{TeV} proton-proton collision event detected by ATLAS on 18 June 2012). This Higgs boson has decayed into two \PZzero bosons, one of which has decayed into a muon and an anti-muon (red lines), and the other has decayed into an electron and a positron (green lines).

The Feynman diagram on the right shows the two incoming protons, each made up of three quarks, which collide.  The Higgs is then formed via a `top loop' resulting from a gluon fusion process in the quark-gluon plasma.

\begin{center}
\begin{minipage}{0.4\textwidth}
\includegraphics[width=\textwidth]{img/higgs.png}
\end{minipage}
\begin{minipage}{0.4\textwidth}
\hfill $\PHiggs\longrightarrow\PZzero\PZzero\longrightarrow\Plepton\APlepton\Plepton\APlepton$
\begin{tikzpicture}
\draw[style={electron}]
(0.5,1.6)--(1.5,1.4);
\draw[style={electron}]
(0.5,1.5)node[left]{\Pproton}--(1.5,1.3);
\draw[style={electron}]
(0.5,1.4)--(1.5,1.2);
\draw[style={electron}]
(0.5,-1.4)--(1.5,-1.2);
\draw[style={electron}]
(0.5,-1.5)node[left]{\Pproton}--(1.5,-1.3);
\draw[style={electron}]
(0.5,-1.6)--(1.5,-1.4);
\draw[style={decorate, draw=black,
        decoration={coil,amplitude=3pt, segment length=4pt}}] %gluon
  (1.5, 1.3) -- node[anchor=south west] {\Pgluon} (2.2,0.5) ;
\draw[style={decorate, draw=black,
        decoration={coil,amplitude=3pt, segment length=4pt}}] %gluon
  (1.5, -1.3) -- node[anchor=north west] {\Pgluon} (2.2,-0.5) ;
\draw[fill=gray] (1.5, 1.3) ellipse (0.2 and 0.3); %proton ellipse
\draw[fill=gray] (1.5,-1.3) ellipse (0.2 and 0.3); %proton ellipse
  \draw[style={electron}]
  (2.2, -0.5) -- node[right] {\Ptop}(2.2, 0.5) ;
  \draw[style={electron}]
  (2.2, 0.5) -- node[above] {\Ptop}(3, 0) ;
  \draw[style={electron}]
  (3, 0) -- node[below] {\Ptop}(2.2, -0.5) ;
\draw[style={dashed,draw=black, postaction={decorate},decoration={markings,mark=at position .55 with {\arrow[draw=black]{stealth}}}}] % HIGGS!!
(3,0)--node[above]{\PHiggs}(4,0);
\draw[style={boson}]
  (4, 0) -- node[anchor=south east] {\PZzero} (5,1) ;
\draw[style={boson}]
  (4, 0) -- node[anchor=north east] {\PZzero} (5,-1) ;
  \draw[style={electron}]
   (5, 1)  -- (6, 1.5) node[right] {\Plepton};
  \draw[style={electron}]
   (5, 1)  -- (6, 0.5) node[right] {\APlepton};
  \draw[style={electron}]
   (5, -1)  --  (6, -0.5) node[right]{\Plepton};
  \draw[style={electron}]
   (5, -1)  -- (6, -1.5){} node[right] {\APlepton};
\end{tikzpicture}
\end{minipage}
\end{center}



\subsubsection{Your answers}

% ANSWERS: 1A, 2B, 3C, 4D, 5A, 6D, 7D, 8B, 9A, 10A, 11D, 12B, 13A, 14D, 15E, 16F, 17B, 18A, 19D, 20D
\noindent\begin{minipage}[t]{0.25\textwidth}
\begin{center}
Event 1/20\\
\includegraphics[width=\textwidth]{img/collider-l5-1.png}
\end{center}
\begin{enumerate}
\item $\PWminus\longrightarrow\Pelectron\Pnue$
\item Background
\item $\PZzero\longrightarrow\Pelectron\APelectron$
\item $\PWminus\longrightarrow\Pmuon\Pnum$
\item $\PHiggs\longrightarrow\Plepton\APlepton\Plepton\APlepton$
\item $\PZzero\longrightarrow\Pmuon\APmuon$
\end{enumerate}
\end{minipage}~
\begin{minipage}[t]{0.25\textwidth}
\begin{center}
Event 2/20\\
\includegraphics[width=\textwidth]{img/collider-l5-2.png}
\end{center}
\begin{enumerate}
\item $\PWminus\longrightarrow\Pelectron\Pnue$
\item Background
\item $\PZzero\longrightarrow\Pelectron\APelectron$
\item $\PWminus\longrightarrow\Pmuon\Pnum$
\item $\PHiggs\longrightarrow\Plepton\APlepton\Plepton\APlepton$
\item $\PZzero\longrightarrow\Pmuon\APmuon$
\end{enumerate}
\end{minipage}~
\begin{minipage}[t]{0.25\textwidth}
\begin{center}
Event 3/20\\
\includegraphics[width=\textwidth]{img/collider-l5-3.png}
\end{center}
\begin{enumerate}
\item $\PWminus\longrightarrow\Pelectron\Pnue$
\item Background
\item $\PZzero\longrightarrow\Pelectron\APelectron$
\item $\PWminus\longrightarrow\Pmuon\Pnum$
\item $\PHiggs\longrightarrow\Plepton\APlepton\Plepton\APlepton$
\item $\PZzero\longrightarrow\Pmuon\APmuon$
\end{enumerate}
\end{minipage}~
\begin{minipage}[t]{0.25\textwidth}
\begin{center}
Event 4/20\\
\includegraphics[width=\textwidth]{img/collider-l5-4.png}
\end{center}
\begin{enumerate}
\item $\PWminus\longrightarrow\Pelectron\Pnue$
\item Background
\item $\PZzero\longrightarrow\Pelectron\APelectron$
\item $\PWminus\longrightarrow\Pmuon\Pnum$
\item $\PHiggs\longrightarrow\Plepton\APlepton\Plepton\APlepton$
\item $\PZzero\longrightarrow\Pmuon\APmuon$
\end{enumerate}
\end{minipage}

\vfill

\noindent\begin{minipage}[t]{0.25\textwidth}
\begin{center}
Event 5/20\\
\includegraphics[width=\textwidth]{img/collider-l5-5.png}
\end{center}
\begin{enumerate}
\item $\PWminus\longrightarrow\Pelectron\Pnue$
\item Background
\item $\PZzero\longrightarrow\Pelectron\APelectron$
\item $\PWminus\longrightarrow\Pmuon\Pnum$
\item $\PHiggs\longrightarrow\Plepton\APlepton\Plepton\APlepton$
\item $\PZzero\longrightarrow\Pmuon\APmuon$
\end{enumerate}
\end{minipage}~
\begin{minipage}[t]{0.25\textwidth}
\begin{center}
Event 6/20\\
\includegraphics[width=\textwidth]{img/collider-l5-6.png}
\end{center}
\begin{enumerate}
\item $\PWminus\longrightarrow\Pelectron\Pnue$
\item Background
\item $\PZzero\longrightarrow\Pelectron\APelectron$
\item $\PWminus\longrightarrow\Pmuon\Pnum$
\item $\PHiggs\longrightarrow\Plepton\APlepton\Plepton\APlepton$
\item $\PZzero\longrightarrow\Pmuon\APmuon$
\end{enumerate}
\end{minipage}~
\begin{minipage}[t]{0.25\textwidth}
\begin{center}
Event 7/20\\
\includegraphics[width=\textwidth]{img/collider-l5-7.png}
\end{center}
\begin{enumerate}
\item $\PWminus\longrightarrow\Pelectron\Pnue$
\item Background
\item $\PZzero\longrightarrow\Pelectron\APelectron$
\item $\PWminus\longrightarrow\Pmuon\Pnum$
\item $\PHiggs\longrightarrow\Plepton\APlepton\Plepton\APlepton$
\item $\PZzero\longrightarrow\Pmuon\APmuon$
\end{enumerate}
\end{minipage}~
\begin{minipage}[t]{0.25\textwidth}
\begin{center}
Event 8/20\\
\includegraphics[width=\textwidth]{img/collider-l5-8.png}
\end{center}
\begin{enumerate}
\item $\PWminus\longrightarrow\Pelectron\Pnue$
\item Background
\item $\PZzero\longrightarrow\Pelectron\APelectron$
\item $\PWminus\longrightarrow\Pmuon\Pnum$
\item $\PHiggs\longrightarrow\Plepton\APlepton\Plepton\APlepton$
\item $\PZzero\longrightarrow\Pmuon\APmuon$
\end{enumerate}
\end{minipage}

\vfill

\noindent\begin{minipage}[t]{0.25\textwidth}
\begin{center}
Event 9/20\\
\includegraphics[width=\textwidth]{img/collider-l5-9.png}
\end{center}
\begin{enumerate}
\item $\PWminus\longrightarrow\Pelectron\Pnue$
\item Background
\item $\PZzero\longrightarrow\Pelectron\APelectron$
\item $\PWminus\longrightarrow\Pmuon\Pnum$
\item $\PHiggs\longrightarrow\Plepton\APlepton\Plepton\APlepton$
\item $\PZzero\longrightarrow\Pmuon\APmuon$
\end{enumerate}
\end{minipage}~
\begin{minipage}[t]{0.25\textwidth}
\begin{center}
Event 10/20\\
\includegraphics[width=\textwidth]{img/collider-l5-10.png}
\end{center}
\begin{enumerate}
\item $\PWminus\longrightarrow\Pelectron\Pnue$
\item Background
\item $\PZzero\longrightarrow\Pelectron\APelectron$
\item $\PWminus\longrightarrow\Pmuon\Pnum$
\item $\PHiggs\longrightarrow\Plepton\APlepton\Plepton\APlepton$
\item $\PZzero\longrightarrow\Pmuon\APmuon$
\end{enumerate}
\end{minipage}~
\begin{minipage}[t]{0.25\textwidth}
\begin{center}
Event 11/20\\
\includegraphics[width=\textwidth]{img/collider-l5-11.png}
\end{center}
\begin{enumerate}
\item $\PWminus\longrightarrow\Pelectron\Pnue$
\item Background
\item $\PZzero\longrightarrow\Pelectron\APelectron$
\item $\PWminus\longrightarrow\Pmuon\Pnum$
\item $\PHiggs\longrightarrow\Plepton\APlepton\Plepton\APlepton$
\item $\PZzero\longrightarrow\Pmuon\APmuon$
\end{enumerate}
\end{minipage}~
\begin{minipage}[t]{0.25\textwidth}
\begin{center}
Event 12/20\\
\includegraphics[width=\textwidth]{img/collider-l5-12.png}
\end{center}
\begin{enumerate}
\item $\PWminus\longrightarrow\Pelectron\Pnue$
\item Background
\item $\PZzero\longrightarrow\Pelectron\APelectron$
\item $\PWminus\longrightarrow\Pmuon\Pnum$
\item $\PHiggs\longrightarrow\Plepton\APlepton\Plepton\APlepton$
\item $\PZzero\longrightarrow\Pmuon\APmuon$
\end{enumerate}
\end{minipage}

\vfill

\noindent\begin{minipage}[t]{0.25\textwidth}
\begin{center}
Event 13/20\\
\includegraphics[width=\textwidth]{img/collider-l5-13.png}
\end{center}
\begin{enumerate}
\item $\PWminus\longrightarrow\Pelectron\Pnue$
\item Background
\item $\PZzero\longrightarrow\Pelectron\APelectron$
\item $\PWminus\longrightarrow\Pmuon\Pnum$
\item $\PHiggs\longrightarrow\Plepton\APlepton\Plepton\APlepton$
\item $\PZzero\longrightarrow\Pmuon\APmuon$
\end{enumerate}
\end{minipage}~
\begin{minipage}[t]{0.25\textwidth}
\begin{center}
Event 14/20\\
\includegraphics[width=\textwidth]{img/collider-l5-14.png}
\end{center}
\begin{enumerate}
\item $\PWminus\longrightarrow\Pelectron\Pnue$
\item Background
\item $\PZzero\longrightarrow\Pelectron\APelectron$
\item $\PWminus\longrightarrow\Pmuon\Pnum$
\item $\PHiggs\longrightarrow\Plepton\APlepton\Plepton\APlepton$
\item $\PZzero\longrightarrow\Pmuon\APmuon$
\end{enumerate}
\end{minipage}~
\begin{minipage}[t]{0.25\textwidth}
\begin{center}
Event 15/20\\
\includegraphics[width=\textwidth]{img/collider-l5-15.png}
\end{center}
\begin{enumerate}
\item $\PWminus\longrightarrow\Pelectron\Pnue$
\item Background
\item $\PZzero\longrightarrow\Pelectron\APelectron$
\item $\PWminus\longrightarrow\Pmuon\Pnum$
\item $\PHiggs\longrightarrow\Plepton\APlepton\Plepton\APlepton$
\item $\PZzero\longrightarrow\Pmuon\APmuon$
\end{enumerate}
\end{minipage}~
\begin{minipage}[t]{0.25\textwidth}
\begin{center}
Event 16/20\\
\includegraphics[width=\textwidth]{img/collider-l5-16.png}
\end{center}
\begin{enumerate}
\item $\PWminus\longrightarrow\Pelectron\Pnue$
\item Background
\item $\PZzero\longrightarrow\Pelectron\APelectron$
\item $\PWminus\longrightarrow\Pmuon\Pnum$
\item $\PHiggs\longrightarrow\Plepton\APlepton\Plepton\APlepton$
\item $\PZzero\longrightarrow\Pmuon\APmuon$
\end{enumerate}
\end{minipage}

\vfill

\noindent\begin{minipage}[t]{0.25\textwidth}
\begin{center}
Event 17/20\\
\includegraphics[width=\textwidth]{img/collider-l5-17.png}
\end{center}
\begin{enumerate}
\item $\PWminus\longrightarrow\Pelectron\Pnue$
\item Background
\item $\PZzero\longrightarrow\Pelectron\APelectron$
\item $\PWminus\longrightarrow\Pmuon\Pnum$
\item $\PHiggs\longrightarrow\Plepton\APlepton\Plepton\APlepton$
\item $\PZzero\longrightarrow\Pmuon\APmuon$
\end{enumerate}
\end{minipage}~
\begin{minipage}[t]{0.25\textwidth}
\begin{center}
Event 18/20\\
\includegraphics[width=\textwidth]{img/collider-l5-18.png}
\end{center}
\begin{enumerate}
\item $\PWminus\longrightarrow\Pelectron\Pnue$
\item Background
\item $\PZzero\longrightarrow\Pelectron\APelectron$
\item $\PWminus\longrightarrow\Pmuon\Pnum$
\item $\PHiggs\longrightarrow\Plepton\APlepton\Plepton\APlepton$
\item $\PZzero\longrightarrow\Pmuon\APmuon$
\end{enumerate}
\end{minipage}~
\begin{minipage}[t]{0.25\textwidth}
\begin{center}
Event 19/20\\
\includegraphics[width=\textwidth]{img/collider-l5-19.png}
\end{center}
\begin{enumerate}
\item $\PWminus\longrightarrow\Pelectron\Pnue$
\item Background
\item $\PZzero\longrightarrow\Pelectron\APelectron$
\item $\PWminus\longrightarrow\Pmuon\Pnum$
\item $\PHiggs\longrightarrow\Plepton\APlepton\Plepton\APlepton$
\item $\PZzero\longrightarrow\Pmuon\APmuon$
\end{enumerate}
\end{minipage}~
\begin{minipage}[t]{0.25\textwidth}
\begin{center}
Event 20/20\\
\includegraphics[width=\textwidth]{img/collider-l5-20.png}
\end{center}
\begin{enumerate}
\item $\PWminus\longrightarrow\Pelectron\Pnue$
\item Background
\item $\PZzero\longrightarrow\Pelectron\APelectron$
\item $\PWminus\longrightarrow\Pmuon\Pnum$
\item $\PHiggs\longrightarrow\Plepton\APlepton\Plepton\APlepton$
\item $\PZzero\longrightarrow\Pmuon\APmuon$
\end{enumerate}
\end{minipage}

\begin{questions}
\setcounter{question}{12}
\question Count up the number of answers for each option, and comment on what you find.

\begin{center}
\renewcommand{\arraystretch}{1.6}
\begin{tabular}{|c|c|c|c|c|c|}
\hline
A & B & C & D & E & F\\
\hline
$\PWminus\longrightarrow\Pelectron\Pnue$ & Background & $\PZzero\longrightarrow\Pelectron\APelectron$ & $\PWminus\longrightarrow\Pmuon\Pnum$ & $\PHiggs\longrightarrow\Plepton\APlepton\Plepton\APlepton$ & $\PZzero\longrightarrow\Pmuon\APmuon$ \\
\hline
& & & & & \\
\hline
\end{tabular}
\renewcommand{\arraystretch}{1}
\end{center}

\fillwithlines{2cm}

\question The `Higgs to 4 leptons' decay could have three final combinations of leptons.  List these.

\fillwithlines{1cm}

\question Which of these would you expect to be more common and why?

\fillwithlines{2cm}

\end{questions}
